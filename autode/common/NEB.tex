\documentclass[10pt]{article}
\usepackage{bm}% bold math
\usepackage{amsmath}

\begin{document}

\subsection{Original NEB}

Nudged elastic band (NEB) approaches to locating transition states are efficient alternatives to evaluating the PES on a uniform grid over some coordinates of interest. The implementation in \emph{autodE} follows that in [\emph{J. Chem. Phys.}, 2000, {\bfseries{113}}, 9978]
\\\\
For an image $i$ in the nudged elastic band
\begin{equation}
\boldsymbol{\tau}_i = 
\begin{cases}
\boldsymbol{\tau}_i^+ &\quad\text{if}\quad V_{i-1} < V_i < V_{i+1} \\
\boldsymbol{\tau}_i^- &\quad\text{if}\quad V_{i+1} < V_i < V_{i-1} \\
\boldsymbol{\tau}_i^+\Delta V_i^{max} + \boldsymbol{\tau}_i^-\Delta V_i^{min} &\quad\text{if}\quad V_{i-1} <  V_{i+1} \\
\boldsymbol{\tau}_i^+\Delta V_i^{min} + \boldsymbol{\tau}_i^-\Delta V_i^{max} &\quad\text{if}\quad V_{i+1} < V_{i-1} \\
\end{cases}
\end{equation}
where
\begin{equation}
\begin{aligned}
\boldsymbol{\tau}_i^+ &= \boldsymbol{x}_{i+1} - \boldsymbol{x}_i \\
\boldsymbol{\tau}_i^- &= \boldsymbol{x}_{i} - \boldsymbol{x}_{i-1}
\end{aligned}
\end{equation}
and
\begin{equation}
\begin{aligned}
\Delta V_i^{max} &= \max(|V_{i+1} - V_i|, |V_{i-1} - V_i|) \\
\Delta V_i^{min} &= \min(|V_{i+1} - V_i|, |V_{i-1} - V_i|)
\end{aligned}
\end{equation}
and $\boldsymbol{x}_i$ are the coordinates of image $i$. The spring force is
\begin{equation}
\boldsymbol{F}^s_i|_{\parallel} = (k_i|\boldsymbol{x}_{i+1} - \boldsymbol{x}_i| - k_{i-1}|\boldsymbol{x}_i - \boldsymbol{x}_{i-1}|) \hat{\boldsymbol{\tau}}_i
\end{equation}
and the total force on the image
\begin{equation}
\boldsymbol{F}_i = \boldsymbol{F}^s_i|_{\parallel} - \nabla V(\boldsymbol{x}_i)|_\perp
\end{equation}
where
\begin{equation}
\nabla V(\boldsymbol{x}_i)|_\perp = \nabla V(\boldsymbol{x}_i) - \nabla V(\boldsymbol{x}_i)\cdot \hat{\boldsymbol{\tau}}_i\hat{\boldsymbol{\tau}}_i
\end{equation}
and finally $\hat{\boldsymbol{\tau}} = \boldsymbol{\tau}_i/|\boldsymbol{\tau}_i|$.
\\\\
\subsection{CI-NEB}

The climbing image (CI) NEB implementation follows that in [\emph{J. Chem. Phys.}, 2000, {\bfseries{113}}, 9901] where after a few iterations the force on the maximum energy image ($m$) is given by

\begin{equation}
\boldsymbol{F}_{m} = -\nabla V(\boldsymbol{x}_m) + 2\nabla V(\boldsymbol{x}_m)\cdot \hat{\boldsymbol{\tau}}_i\hat{\boldsymbol{\tau}}_i
\end{equation}

which is the force due to the potential along the band being inverted.


\end{document}